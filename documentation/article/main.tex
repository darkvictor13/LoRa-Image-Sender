\documentclass[
% -- opções da classe memoir --
article,			% indica que é um artigo acadêmico
12pt,				% tamanho da fonte
oneside,			% para impressão apenas no recto. Oposto a twoside
a4paper,			% tamanho do papel. 
english,			% idioma adicional para hifenização
brazil,				% o último idioma é o principal do documento
sumario=tradicional
]{abntex2}
% ---
% PACOTES
% ---
\usepackage{lmodern} % Usa a fonte Latin Modern
\usepackage[T1]{fontenc}% Selecao de codigos de fonte.
\usepackage[utf8]{inputenc}% Codificacao do documento (conversão automática dos acentos)
\usepackage{indentfirst}% Indenta o primeiro parágrafo de cada seção.
\usepackage{nomencl} % Lista de simbolos
\usepackage{color}% Controle das cores
\usepackage{graphicx}% Inclusão de gráficos
\usepackage{microtype}% para melhorias de justificação
\usepackage{gensymb}
\usepackage{caption}
\usepackage{booktabs}
\usepackage{footnote}
\usepackage{listings}

\lstset{frame=tb,
  language=C++,
  aboveskip=3mm,
  belowskip=3mm,
  showstringspaces=false,
  columns=flexible,
  basicstyle={\small\ttfamily},
  numbers=left,
  numberstyle=\tiny\color{gray},
  keywordstyle=\color{blue},
  commentstyle=\color{dkgreen},
  stringstyle=\color{mauve},
  breaklines=true,
  breakatwhitespace=true,
  tabsize=4
}

% ---
% Pacotes adicionais, usados apenas no âmbito do Modelo Canônico do abnteX2
% ---
% ---
% Pacotes de citações
% ---
\usepackage[brazilian,hyperpageref]{backref}	 % Paginas com as citações na bibl
%\usepackage[alf,abnt-emphasize=bf]{abntex2cite}	% Citações padrão ABNT
\usepackage[num,overcite,abnt-emphasize=bf]{abntex2cite}	% Citações padrão ABNT
%\citebrackets()
\citebrackets[]
% ---

% ---
% Configurações do pacote backref
% Usado sem a opção hyperpageref de backref
\renewcommand{\backrefpagesname}{Citado na(s) página(s):~}
% Texto padrão antes do número das páginas
\renewcommand{\backref}{}
% Define os textos da citação
\renewcommand*{\backrefalt}[4]{
  \ifcase #1 %
    Nenhuma citação no texto.%
    \or
    Citado na página #2.%
  \else
    Citado nas páginas #2.%
\fi}%
% ---

\graphicspath{{./images/}}

% --- Informações de dados para CAPA e FOLHA DE ROSTO ---
\titulo{Envio de imagens via LoRa}
\tituloestrangeiro{title}


\autor{Victor Emanuel Almeida \and Marco A. Guerra}
%\autor{
  %UNIOESTE\thanks{``Universidade Estadual do Oeste do Paraná, Foz do Iguaçu, Brasil.'' \url{http://www.unioeste.br/}} 
  %\\[0.5cm]\\
  %Victor Emanuel Almeida\thanks{``Estudante de Ciência da Computação na Universidade Estadual do Oeste do Paraná (UNIOESTE), campus de Foz do Iguaçu-PR, Brasil.''\url{http://www.unioeste.br/}}
%}

\local{FOZ DO IGUAÇU}
\data{\today}

\instituicao{%
\par
Universidade do Oeste do Paraná 
\par
UNIOESTE
}
\instituicao{Curso de Ciência da Computação, da Universidade Estadual do Oeste do Paraná (UNIOESTE), Campus Foz do Iguaçu-PR, Brasil}

\preambulo{Escrever Preâmbulo}

\tipotrabalho{Trabalho Acadêmico}
% ---

% ---
% Configurações de aparência do PDF final

% alterando o aspecto da cor azul
\definecolor{blue}{RGB}{41,5,195}

% informações do PDF
\makeatletter
\hypersetup{
  %pagebackref=true,
  pdftitle={\@title}, 
  pdfauthor={\@author},
  pdfsubject={software livre},
  pdfcreator={\@author},
  pdfkeywords={software livre},
  colorlinks=true,% false: boxed links; true: colored links
  linkcolor=black,% color of internal links
  citecolor=blue,% color of links to bibliography
  filecolor=magenta,% color of file links
  urlcolor=blue,
  bookmarksdepth=4
}
\makeatother
% --- 

% ---
% compila o indice
% ---
\makeindex
% ---

% ---
% Altera as margens padrões
% ---
\setlrmarginsandblock{3cm}{2cm}{*}
\setulmarginsandblock{3cm}{2cm}{*}
\checkandfixthelayout
% ---

% --- 
% Espaçamentos entre linhas e parágrafos 
% --- 

% O tamanho do parágrafo é dado por:
\setlength{\parindent}{1.25cm}
%\setlength{\parindent}{1.5\lineheight}

% Controle do espaçamento entre um parágrafo e outro:
%\setlength{\parskip}{0.2cm}  % tente também \onelineskip
\setlength{\parskip}{\onelineskip}

% Espaçamento simples
\SingleSpacing

% ----
% Início do documento
% ----
\begin{document}
%\pagenumbering{roman}
% Seleciona o idioma do documento (conforme pacotes do babel)
%\selectlanguage{english}
\selectlanguage{brazil}

% Retira espaço extra obsoleto entre as frases.
\frenchspacing 

% ----------------------------------------------------------
% ELEMENTOS PRÉ-TEXTUAIS
% ----------------------------------------------------------

%---
%
% Se desejar escrever o artigo em duas colunas, descomente a linha abaixo
% e a linha com o texto ``FIM DE ARTIGO EM DUAS COLUNAS''.
%\twocolumn[    		% INICIO DE ARTIGO EM DUAS COLUNAS
%
%---

% página de titulo principal (obrigatório)
%\imprimircapa
%\begin{center}
\maketitle
%{\centered\maketitle\imprimirlocal\imprimirinstituicao}
%\imprimirtitulo
%\vspace{.5cm}

%\imprimirinstituicao
%\end{center}

% titulo em outro idioma (opcional)

% resumo em português
\begin{resumoumacoluna}
    % De 100 a 250 palavras
    % frases curtas
    % tem que falar de:
        % objetivo
        % matérias e métodos
        % resultados
        % concluir
  \textbf{Palavras-chave}: Agricultura de precisão, Internet das coisas, Sensores.
  \vspace{\onelineskip}
  \noindent
\end{resumoumacoluna}

% resumo em inglês
\renewcommand{\resumoname}{Abstract}
\begin{resumoumacoluna}
  \begin{otherlanguage*}{english}
    Escrever resumo
    \vspace{\onelineskip}
    \noindent

    \textbf{Keywords}: LoRa, Redes LPWAN, Envio de imagens
  \end{otherlanguage*}  
\end{resumoumacoluna}

%]  				% FIM DE ARTIGO EM DUAS COLUNAS
% ---

% ----------------------------------------------------------
% ELEMENTOS TEXTUAIS
% ----------------------------------------------------------
\textual
% ----------------------------------------------------------
% Introdução
% ----------------------------------------------------------
\section{Introdução}
Introdução

\section{Materiais e métodos}\label{Materiais e métodos}

\subsection{Protocolos utilizados}\label{Protocolos utilizados}
\subsubsection{Stop and Wait}\label{Stop and Wait}

\subsubsection{Detecção de erros: CRC}\label{CRC}

Falar dos algoritmos de Detecção de erro (CRC) e correção de erros (Hamming)

\begin{lstlisting}[title=Algoritmo CRC 16 bits]
static const uint16_t POLY = 0xA001;
static const uint16_t INIT = 0xC181;

uint16_t computeCRC(uint8_t* data_in, uint16_t length) {
    uint16_t i;
    uint8_t bitbang, j;
    uint16_t crc_calc = INIT;

    for(i = 0; i < length; i++) {
        crc_calc ^= (((uint16_t)data_in[i]) & 0x00FF);
        for(j = 0; j < 8; j++) {
            bitbang = crc_calc;
            crc_calc >>= 1;

            if(bitbang & 1) {
                crc_calc ^= POLY;
            }
        }
    }
    return (crc_calc & 0xFFFF);
}
\end{lstlisting}

\section{Estrutura dos pacotes}\label{Estrutura dos pacotes}

% ----------------------------------------------------------
% ELEMENTOS PÓS-TEXTUAIS
% ----------------------------------------------------------
\postextual

% ----------------------------------------------------------
% Referências bibliográficas
% ----------------------------------------------------------
%\cleardoublepage
\bibliography{ref}

\end{document}
